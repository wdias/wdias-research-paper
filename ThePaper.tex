\documentclass[conference]{IEEEtran}
\IEEEoverridecommandlockouts
% The preceding line is only needed to identify funding in the first footnote. If that is unneeded, please comment it out.

\usepackage[acronym]{glossaries}
\usepackage{cite}
\usepackage{amsmath,amssymb,amsfonts}
\usepackage{algorithmic}
\usepackage{graphicx}
\usepackage{textcomp}
\usepackage{xcolor}

\usepackage{listings}
% \usepackage{biblatex}
\usepackage{adjustbox}
\usepackage{tabulary}

\usepackage{todonotes}
\newcommand{\db}[1]{\textcolor{blue!40}{#1}}
\newcommand{\dbc}[1]{\todo[author=Dilum, inline, color=blue!40]{#1}}
\newcommand{\gk}[1]{\textcolor{orange}{#1}}
\newcommand{\gkc}[1]{\todo[author=Gihan, inline, color=green!40]{#1}}

\usepackage[hidelinks]{hyperref}
\usepackage{cleveref}

\crefformat{section}{#2Sec.~#1#3}
\crefformat{figure}{#2Fig.~#1#3}
\crefformat{table}{#2Table~#1#3}

% \addbibresource{mendeley.bib}

\definecolor{codegreen}{rgb}{0,0.6,0}
\definecolor{backcolour}{rgb}{0.95,0.95,0.92}
\lstdefinestyle{code-style}{
  backgroundcolor=\color{backcolour}, commentstyle=\color{codegreen},
  basicstyle=\ttfamily\footnotesize,
  breakatwhitespace=false,
  breaklines=true,
  captionpos=b,
  keepspaces=true,
  numbers=left,
  numbersep=1pt,
  showspaces=false,
  showstringspaces=false,
  showtabs=false,
  tabsize=1
}
\lstset{style=code-style}

\def\BibTeX{{\rm B\kern-.05em{\sc i\kern-.025em b}\kern-.08em
    T\kern-.1667em\lower.7ex\hbox{E}\kern-.125emX}}
    
\begin{document}

\title{WDIAS: A Microservices-Based Weather Data Integration and Assimilation System\\
}

\author{
%\IEEEauthorblockN{Gihan Karunarathne}
%\IEEEauthorblockA{\textit{Dept. Computer Science and Engineering} \\
%\textit{University of Moratuwa}\\
%Katubedda, Sri Lanka \\
%gihan.09@cse.mrt.ac.lk}
%\and
%\IEEEauthorblockN{H.M.N. Dilum Bandara}
%\IEEEauthorblockA{\textit{Dept. Computer Science and Engineering} \\
%\textit{University of Moratuwa}\\
%Katubedda, Sri Lanka \\
%dilumb@cse.mrt.ac.lk}
%\and
%\IEEEauthorblockN{Srikantha Herath}
%\IEEEauthorblockA{\textit{Center for Urban Water, Sri Lanka} \\
%Battaramulla, Sri Lanka \\
%admin@curwsl.org}
}


\maketitle

%%%%%%%%%%%%%%%%%%%%%%%%%%%%%%%%%%%%%%%%%%%%%%%%%%%%%%%%%%%%%%%%%%%%%%%%%%%%%%%%
\begin{abstract}
Numerical Weather Models (NWMs) utilize data from diverse sources such as automated weather stations, radars, and satellite images. Such multimodal data need to be transcoded into a NWM compatible format before use. Moreover, the data integration system's response time needs to be relatively low to reduce the time to forecast weather events like hurricanes and flash floods. Furthermore, the resulting data need to be accessed by many researchers and third-party applications. Existing weather data integration systems are based on monolithic or client-server architectures, and are proprietary or closed source. Hence, they are not only expensive to operate in an era of cloud computing, but also challenging to customize for regions with different weather patterns. In this paper, we present a Weather Data Integration and Assimilation System (WDIAS) that uses microservices architecture and container orchestration to achieve high scalability, availability, and low-cost operation. WDIAS provides a modular architecture to integrate data from different sources, enforce data quality controls, export data into different formats, and extend the functionality by adding new modules. Using a synthetic workload and an experimental setup on a public cloud, we demonstrate that WDIAS can handle 300 requests per second with relatively low latency.
\end{abstract}

%\dbc{Removing author details as review process is double blind.}

\begin{IEEEkeywords}
Cloud computing, data assimilation, data integration, microservice, weather
\end{IEEEkeywords}

%%%%%%%%%%%%%%%%%%%%%%%%%%%%%%%%%%%%%%%%%%%%%%%%%%%%%%%%%%%%%%%%%%%%%%%%%%%%%%%%
\section{Introduction}
\label{pse:Introduction}

Weather prediction is essential to reduce impact due to natural disasters and to manage natural resources like water effectively. To enhance the accuracy of weather predictions, it is necessary to provide reliable and detailed weather data as inputs to \acrfull{nwm}s. Before feeding multimodal data collected from different sources into a NWM, they need to be converted to a format that can be ingested by the NWM. Further, the velocity and volume of data vary from event-driven bulk uploads to streaming data. Such data need to be ingested, preprocessed, converted, stored, and accessed with minimal latency, especially when forecasting time-sensitive weather events like hurricanes and flash floods. Processed weather data may also need to be accessed by many third-party applications and researchers. For example, logistics and insurance companies could use weather data for planning purposes or set their service fees or premiums based on anticipated weather patterns.

Several data integration systems are designed for weather-related use cases. For example, \acrshort{fews} \cite{Werner2013TheSystem} uses a model-centric approach where it handles the forecasting process as a combination of data modeling steps and data transformation algorithms. It creates forecasting workflows by integrating new models and algorithms based on the open model integration framework \cite{Kokkonen2003InterfacingXML}. \acrshort{fews} follows a common data model and enforces data access via a well-defined interface. While such a design leads to efficient data access and management, it requires all NWMs and connected systems to comply with the common data model, tightly couples the models, and serializes the model execution. 
\acrfull{lead} \cite{Droegemeier2005Service-OrientedWeather} provides dynamic workflow orchestration and data management based on Web-services. It provides several service layers based on the \acrfull{soa}, exposes services via a drag-and-drop interface to create new workflows, and aggregates computing resources into a pool. This design results in a high performing, customizable, scalable, and resource-efficient weather analysis framework. 
\acrfull{dias} \cite{Kawasaki2018DataReduction} is another related system that provides efficient data storage, data quality controls, metadata management, and data sharing. \acrfull{madis} \cite{Macdermaid2005ArchitectureP2.39} enforces a common data format and provides role-based data access. 
However, \acrshort{dias} and \acrshort{madis} do not support workflows. All these systems are based on monolithic or client-service architectures and are platform dependent. Thus, they cannot gain the full performance and economic benefits of contemporary technologies such as cloud computing. Elasticity is essential in weather-related workload as closed-loop weather monitoring systems and multiple NWMs models are fully utilized during only a significant weather event. Further, these systems are either proprietary or closed source, limiting the wider adoption and customization. For example, it is challenging to customize them for an island like Sri Lanka with different weather patterns.

In this paper, we present a cloud-based \acrfull{wdias} that supports data integration, assimilation, and dissemination. We designed the proposed system based on the microservices architecture to make it modular, extensible, scalable, and highly available compared to related systems. \acrshort{wdias} can integrate timeseries data in text and binary formats from different sources, export data into different formats, and add extension modules for custom data preprocessing, transformations, and quality controls. The proposed platform further uses container orchestration to simplify application management and gain cost savings through the auto-scaling of cloud computing resources. 
Using an experimental setup on a public cloud and synthetic workload derived from weather use cases and real weather stations, we demonstrated that \acrshort{wdias} can handle 300 requests (with different data types) per second with good latency characteristics. Also, the system is capable of running on a broad range of computing resources ranging from a few CPUs to hundreds of CPUs, as well as show good elasticity against varying workloads.

The rest of the paper is organized as follows: The architecture of \acrshort{wdias} is presented in \cref{pse:wdias_architecture}. In \cref{pse:performance_analysis}, we present the performance analysis. Concluding remarks are presented in \cref{pse:summary}.

%%%%%%%%%%%%%%%%%%%%%%%%%%%%%%%%%%%%%%%%%%%%%%%%%%%%%%%%%%%%%%%%%%%%%%%%%%%%%%%%

\begin{figure}[!tb]
\centerline{\includegraphics[width=0.5\textwidth]{images/weather_data_system_components_p1.pdf}}
\caption{Components of a weather data integration and assimilation system.}
\label{pfi:wdia_components}
\end{figure}

%%%%%%%%%%%%%%%%%%%%%%%%%%%%%%%%%%%%%%%%%%%%%%%%%%%%%%%%%%%%%%%%%%%%%%%%%%%%%%%%
\section{Proposed Architecture}
\label{pse:wdias_architecture}

%%%%%%%%%%%%%%%%%%%%%%%%%%%%%%%%%%%%%%%%%%%%%%%%
\subsection{WDIAS Microservices Architecture}
\label{psubse:wdias_microservices}

\cref{pfi:wdia_components} shows the essential functions of a weather data integration and assimilation system such as integration, assimilation, and dissemination. A typical system should be capable of integrating multimodal and multidimensional spatial and temporal weather data from diverse sources such as satellites, radars, and high-end and low-end weather stations. In assimilation, the system not only stores the data but also provides them to various weather models in their desired data formats. In the dissemination step, both the captured weather data and processed data from weather models are shared with external systems and users based on their data subscriptions, timeseries metadata and geographic location-based queries, and access rights. \acrshort{wdias} supports all three of these modules.

We designed \acrshort{wdias} based on the microservices architecture, as it enables resilient and flexible services that can be independently scaled while achieving high-availability \cite{LewisMicroservices}. Such a design allows each module to be offered as a separate microservice where each module works on a separate piece of bulk or stream data. Microservices further enables modules to be written in any language and dynamically added to the system to extend its functionality. We implemented microservices as containerized applications. \emph{Application containerization} is an alternative to virtualization where a software and all its dependencies are encapsulated/grouped into a container that can execute uniformly and consistently in any infrastructure \cite{IBMContainerizationExplained}. Containerization has emerged as an industry standard for packaging software into standardized units for development, shipment, and deployment. Therefore, the use of containerization simplifies the deployment and maintenance of \acrshort{wdias} services on a cloud platform resulting in simplified administration and cost savings.

\begin{figure}[!tb]
\centerline{\includegraphics[width=0.5\textwidth]{images/separation_microservices-p1.pdf}}
\caption{Separation of \acrshort{wdias} microservices.}
\label{pfi:microservice_separation}
\end{figure}

\cref{pfi:microservice_separation} shows the separation of \acrshort{wdias}'s modules into the microservices where each circle represents a microservice. All microservices communicate using a RESTful API. Import modules are shown on the left of figure while export modules are shown on the right. \acrshort{wdias} supports importing
and exporting scalar, vector, and grid timeseries data in multiple formats such as JSON, CSV, NetCDF, and ASCII.
Further, users can add new import and export microservices to support different data formats and ingest/egress methods. Each \emph{import microservice} coverts data to a common scalar, vector, or grid format, and forwards them to the relevant data adapter module which stores the data. For example, \emph{scalar} and \emph{vector} adapters support data in JSON format. Before feeding grid data into the grid adapter, import grid microservices need to transform data into \acrlong{netcdf} (\acrshort{netcdf}) format. Each adapter has its database instance optimized to store a specific type of data. For example, timeseries metadata is stored in a relational database for faster querying. It is also cached in an in-memory database to provide low-latency access. \emph{Export microservices} provide data to weather models, users, and third-party data consumers while converting them to the desired format. \emph{Extension} modules perform data preprocessing. \emph{Extension adapter} stores metadata required to trigger an extension. While the \emph{extension handler} triggers an extension when a new piece of data arrives, the \emph{extension scheduler} triggers an extension based on a predefined schedule. By implementing microservices as containerized applications, we further simplify deployment of modules and enhance platform independence. Further, \acrshort{wdias} uses \acrfull{k8s} \cite{LinuxFoundationProduction-GradeKubernetes} container orchestration platform to auto-scale microservice instances as the workloads vary.

% \begin{figure}[t!]
% \centerline{\includegraphics[width=0.5\textwidth]{images/microservice_architecture-handle_on_async-p1.pdf}}
% \caption{Asynchronously handling requests with microservices.}
% \label{pfi:microservice_architecture_async}
% \end{figure}

Each piece of data is tagged with a unique identifier (ID), which is used by all microservices to handle the data. A timeseries is considered as a single piece of data. Ingestion of a large piece of data, such as radar data in \acrshort{netcdf} format, is handled asynchronously. When a request with a large piece of data is received, rather than blocking the request, a unique ID is returned to the caller. This ID can be used later to verify the successful processing of data.

% \db{As shown in \cref{pfi:microservice_architecture_async}, we use a Status Handler to manage the status of asynchronous requests. For example, when an ingestion module receives a large piece of data it is stored on the module. Then the module publishes an event to another service to process it. Once the successful processing of the data is verified, Status Handler updates the system status.}


%%%%%%%%%%%%%%%%%%%%%%%%%%%%%%%%%%%%%%%%%%%%%%%%
\subsection{Database Structure}
\label{psubse:wdias_database}

Timeseries are pervasive in the weather domain from observations to forecasting. Thus, \acrshort{wdias} is designed to be data-centric with a particular focus on timeseries data. Data points in a weather-related timeseries can vary from scalar (0D), vector (1D), grid (2D), to polygon (2D) values. We used the following list of metadata to identify a timeseries uniquely:

\begin{itemize}
    \item \texttt{Module ID} -- is the source of data, e.g., weather station ID.
    \item \texttt{Location} -- that the data corresponds to, e.g., point locations and regular or irregular grid locations.
    \item \texttt{Value Type} -- indicates whether data consists of scalar, vector, grid, or polygon values.
    \item \texttt{Parameters} -- indicate the ambient measurements such as temperature, pressure, wind speed, and wind direction.
    \item \texttt{Timeseries Type} -- indicates whether the data are historical or a forecast of a model.
    \item \texttt{Time Step} -- indicates whether the sampling interval is uniform or not.
\end{itemize}

While a scalar data point consists of a single value, a vector data point consists of two values (e.g., magnitude and the direction of a vector). Whereas a grid data point consists of multiple values. Therefore, to manipulate data efficiently, as seen in \cref{pfi:database_structure} we used a combination of relational, NoSQL, timeseries, and in-memory databases, as well as files.
To isolate performance and faults, we handle scalar and vector values using two different microservices referred to as \textit{adapter-scalar} and \textit{adapter-vector} (see \cref{pfi:microservice_separation}). We used InfluxDB to manage scalar and vector timeseries data due to its high performance and scalability in handling timeseries data. We used \acrshort{netcdf} files to store grid data as it is the de-facto standard for creating, accessing, and sharing of array-oriented scientific data. To resolve a query targeting timeseries metadata, we need to identify first the source and type of data, as well as where they are stored and what time or geographic range to retrieve. Hence, we used a relational database to manipulate timeseries metadata with low latency. We further used Redis in-memory database to enable low-latency query resolution by caching data stored in \textit{adapter-metadata} and \textit{adapter-extension} microservices. Further, spacial indexing (aka., geo-indexing) is needed to support queries based on the geographic location of data, which are one of the hardest queries to resolve. Therefore, we used MongoDB, a document-oriented database, to store timeseries metadata as documents for fast retrieval and index searching, as well as to provide geo-indexing for location-based queries.

\begin{figure}[!tb]
\centerline{\includegraphics[width=0.5\textwidth]{images/wdias_database_structure_p1.pdf}}
\caption{\acrshort{wdias} database structure.}
\label{pfi:database_structure}
\end{figure}

\acrshort{wdias} supports timeseries and geographic metadata based queries. For example, a query looking for a timeseries in a given area (specified as a polygon in \emph{geoJson} \cite{InternetEngineeringTaskForceGeoJSON} format) could be resolved using geo-indexing. It also supports querying timeseries with a given weather parameter and area of interest, as well as locations within an area of a polygon. While resolving queries made by third-party users, query results include only the unique ID and metadata of timeseries. Subsequently, the user can retrieve the data for the given period by submitting this ID to the \emph{export module}.


%%%%%%%%%%%%%%%%%%%%%%%%%%%%%%%%%%%%%%%%%%%%%%%%
\subsection{Data Preprocessing}
\label{psubse:data_preprocessing}

\acrshort{wdias} provides data preprocessing capabilities using extension modules. As seen in \cref{pfi:summary_weather_data_preprocessing}, each extension could be considered as a mathematical function that takes \texttt{p} timeseries variables as input and output \texttt{n} timeseries variables. As seen on the bottom half of figure, these functions are typically used for data quality checks, interpolation (both serial and spatial), and transformation. To further extend the function at run time without creating a new module, the \textit{bind constant} can be set at the time of creating a trigger for an extension. For example, such features are needed as we may need to generate two different timeseries by sub-sampling 1-minute data to 15 minutes and one hour. While this increases the data stored by \acrshort{wdias}, it reduces the computational time as NWMs can readily use the data without transforming them at the run time. 

Listing 1 shows the format of a request made to update an extension at the time of triggering it. Each extension trigger needs to have a unique ID (line 2). Then the user can define the extension type and the microservice responsible of processing it (see lines 3 and 4). Using lines 5-12, the user can map timeseries into input variable \emph{x} and output variable \emph{y} as seen in \cref{pfi:summary_weather_data_preprocessing}. Finally, the user can also define when to trigger the extension (e.g., \textit{OnChange} or \textit{OnTime}) and any options such as bind constant. Using different update requests, \acrshort{wdias} can create multiple triggers for an extension. When the trigger is set as \emph{OnChange}, the \emph{extension handler} triggers the registered extension when new data are ingested into the timeseries. Whereas the \emph{extension scheduler} triggers the registered extension when the trigger is set as \emph{OnTime}. 


\begin{figure}[!tb]
\centerline{\includegraphics[width=0.4\textwidth]{images/summary_weather_data_preprocessing_p1.pdf}}
\caption{Functional approach to weather data preprocessing.}
\label{pfi:summary_weather_data_preprocessing}
\end{figure}

\begin{minipage}{0.47\textwidth}
\begin{lstlisting}[language=Python, caption=Format of a request made to update an extension., label=pli:extension_triggers]
{
    "extensionId": "", //Trigger unique ID
    "extension": "Interpolation | Transformation | Validation",
    "function": "", //Mapping microservice
    "variables": [ //Map timeseries to variables
        {
            "variableId": "",
            "metadata/metadtaIds": {...}
        } 
    ],
    "inputVariables": [], //Input timeseries
    "outputVariables": [], //Output timeseries
    "trigger": [ // When to trigger the function
        {}
            "trigger_type": "OnChange | OnTime",
            "trigger_on": []
        }
    ],
    "options": { //Run time bind constant data
    }
}
\end{lstlisting}
\end{minipage}


%%%%%%%%%%%%%%%%%%%%%%%%%%%%%%%%%%%%%%%%%%%%%%%%%%%%%%%%%%%%%%%%%%%%%%%%%%%%%%%%
\section{Performance Analysis}
\label{pse:performance_analysis}

%%%%%%%%%%%%%%%%%%%%%%%%%%%%%%%%%%%%%%%%%%%%%%%%
\subsection{Experimental Setup}
\label{psubse:experimental_setup}

As a weather data integration and assimilation system is expected to handle a variety of data sources and types, we created a synthetic workload consisting of 70\% scalar, 20\% vector, and 10\% grid values. 
We created 1,000 timeseries using 30-day traces from five weather stations of the \acrlong{curw}. Locations of timeseries were set based on Google's Countries dataset with 250 locations \cite{GoogleGoogleCounties}. 
For each location, we created four timeseries by varying the source of data (i.e., \texttt{Module ID}) to create a total of 1,000 timeseries. Weather data for a day were included in a single request, and a date counter was used to keep track of the day. For each day, scalar timeseries randomly picked precipitation data from one of the five weather stations. Once the 30 days are reached, we went back to the first date of the trace and continued again. Similarly, vector timeseries randomly picked wind speed and direction data from one of the weather stations. To create grid data, we considered a 2D regular grid of $120\times139$ points. 
We created 100 grid timeseries using 30-day traces from simulated water levels produced an ASCII grid file for each simulation.
Therefore, while scalar and vector data points contained a one and two values for a timestamp, a grid data point contained $120\times139$ values. Thus, grid data are much more difficult to handle compared to scalar and vector data.

Load testing was performed using Apache JMeter with distributed testing feature, which use multiple workload generators to create a high throughput workload. For both insert and retrieve requests, we varied the request size by varying the number of data points per request. For example, we insert or retrieve data for an entire day at 60, 30, and 15-min resolutions. For a given request size, we issued requests at different rates such as 10, 50, 100, 200, and 300 requests per second. For each request size, we ran a separate load test while increasing the request rate from 10 to 300 within 30 min. After reaching the highest request rate, we also measured how the system would gradually release the resources when there is no load on the system. We also issued a set of requests to download a timeseries nearest to a given \emph{location}. This request invokes a geo-search query to identify all nearby timeseries datasets; hence, captures the performance of geo-indexing solution.

\acrshort{wdias} was deployed using \acrfull{eks} container orchestration platform. \cref{ptab:aws_eks_nodes} shows the list of nodes used in the experimental setup. The \emph{metadata}, \emph{query}, \emph{status}, and \emph{extension} microservices were scheduled on the node labeled as the \emph{core}. The \emph{grid adapter}, as well as grid data import and export microservices were scheduled on the \emph{grid node} as they frequently communicate with each other. Similarly, \emph{scalar} and \emph{vector} adapters, as well as scalar and vector import and export services were deployed on the \emph{scalar} node. JMeter test server was hosted on the \emph{test} node. No data preprocessing modules were deployed during the performance analysis.


\begin{table}[tb!]
\caption{Configuration of \acrshort{eks} nodes.}
\begin{center}
\begin{adjustbox}{width=0.45\textwidth}
\begin{tabular}{|l|r|r|r|r|l|}
\hline
\textbf{Node Label} & \textbf{vCPUs} & \textbf{RAM (GB)} & \textbf{Storage (GB)} & \textbf{Quantity} & \textbf{EC2 Name} \\ \hline
core & 16 & 32 & 15 & 1 & c5.4xlarge \\ \hline
grid & 8 & 16 & 25 & 1 & c5.2xlarge \\ \hline
scalar & 8 & 16 & 20 & 1 & c5.2xlarge \\ \hline
test & 4 & 10.5 & 5 & 1 & c5n.xlarge \\ \hline
\end{tabular}
\end{adjustbox}
\label{ptab:aws_eks_nodes}
\end{center}
\end{table}


%%%%%%%%%%%%%%%%%%%%%%%%%%%%%%%%%%%%%%%%%%%%%%%%
\subsection{Performance Evaluation}
\label{psubse:performance_evaluation}

\cref{ptab:obs_all_60_min_summary_throughput} to \cref{ptab:obs_all_15_min_summary_throughput} show the throughput (measured as \acrlong{rps} (\acrshort{rps})), latency, and errors under three different data resolutions. Whereas \cref{ptab:obs_all_auto_15_min_summary_throughput} shows the performance when the container auto-scaling was enabled. It can be seen that each test resulted in approximately 310K requests within 30 minutes. Avg., 90\%, and STD indicate the average, standard deviation, and 90\% percentile for latency. Other than some grid timeseries inserts, all other operations succeed with no errors.

As we increase the request size, it can be seen that the average and 90\% percentile latency to insert scalar and vector timeseries increased by 2 ms and 10 ms, respectively (see  \cref{ptab:obs_all_60_min_summary_throughput} to \cref{ptab:obs_all_15_min_summary_throughput}). Still, \acrshort{wdias} was able to accept an average of 40.5 \acrshort{rps}. However, when retrieving the scalar and vector timeseries data with the 15-min resolution, latency got doubled compared to the 30-min resolution test case. This was due to the doubling of data in insert and retrieve requests where both the requests were handled by the same database. However, there was no noticeable reduction in RPS. Therefore, even though the data volume increased by four times, \acrshort{wdias} performance degradation is sub-linear. 


\begin{table}[tb!]
\centering
\caption{Performance while processing 60-min data requests.}
\begin{adjustbox}{width=0.45\textwidth}
% \footnotesize
\begin{tabular}{|l|r|r|r|r|r|r|}
\hline
\textbf{Request Type} & \textbf{\# Samples} & \textbf{RPS} & \textbf{Avg.} & \textbf{90\%} & \textbf{STD} & \textbf{Error \%} \\ \hline
Insert Timeseries & 71826 & 40.5 & 28 & 31 & 58.74 & 0.00\% \\ \hline
Retrieve Timeseries & 71796 & 40.7 & 8 & 10 & 4.18 & 0.00\% \\ \hline
Insert Grid & 7982 & 4.5 & 23 & 26 & 4.23 & 0.06\% \\ \hline
Retrieve Grid & 7979 & 4.5 & 68 & 75 & 10.11 & 0.00\% \\ \hline
Query Location & 71804 & 40.5 & 3 & 3 & 1.52 & 0.00\% \\ \hline
\textbf{TOTAL} & 311182 & 175.4 & 127 & 503 & 207.80 & 0.00\% \\ \hline
%\multicolumn{4}{l}{$^{\mathrm{a}}$S.D.: Standard Deviation}{$^{\mathrm{b}}$90\%: 90\% percentile}
\end{tabular}
\end{adjustbox}
\label{ptab:obs_all_60_min_summary_throughput}
\end{table}


\begin{table}[tb!]
\caption{Performance while processing 30-min data requests.}
\begin{center}
\begin{adjustbox}{width=0.45\textwidth}
% \footnotesize
\begin{tabular}{|l|r|r|r|r|r|r|}
\hline
\textbf{Request Type} & \textbf{\# Samples} & \textbf{RPS} & \textbf{Avg.} & \textbf{90\%} & \textbf{STD} & \textbf{Error \%} \\ \hline
Insert Timeseries & 71759 & 40.5 & 29 & 32 & 50.97 & 0.00\% \\ \hline
Retrieve Timeseries & 71730 & 40.6 & 9 & 10 & 6.04 & 0.00\% \\ \hline
Insert Grid & 7972 & 4.5 & 44 & 49 & 8.17 & 0.08\% \\ \hline
Retrieve Grid & 7971 & 4.5 & 81 & 93 & 15.15 & 0.00\% \\ \hline
Query Location & 71734 & 40.5 & 3 & 3 & 1.90 & 0.00\% \\ \hline
TOTAL & 310878 & 175.3 & 129 & 0 & 207.10 & 0.00\% \\ \hline
%\multicolumn{4}{l}{$^{\mathrm{a}}$S.D.: Standard Deviation}{$^{\mathrm{b}}$90\%: 90\% percentile}
\end{tabular}
\end{adjustbox}
\label{ptab:obs_all_30_min_summary_throughput}
\end{center}
\end{table}

Latency to insert grid timeseries data is relatively low compared to scalar or vector inserts as grid data are handled asynchronously, and before processing data, the request handler store the grid files upload via a single thread. When the number of grid files doubled, latency also doubled, as the request handler had to store twice as much data. Yet, \acrshort{wdias} was able to handle all the incoming requests. The error rate increased as some grid data writes were queued up while waiting to be written as \acrshort{netcdf} files. Some of these requests took more than 10 second, which was the JMeter timeout we set to avoid requests from piling up. Grid timeseries insert and retrieve requests require manipulation of \acrshort{netcdf} files, which is a computationally and IO intensive process. Hence, grid data operations have a much higher latency compared to scalar and vector requests.

\begin{table}[tb!]
\caption{Performance while processing 15-min data requests.}
\begin{center}
\begin{adjustbox}{width=0.45\textwidth}
% \footnotesize
\begin{tabular}{|l|r|r|r|r|r|r|}
\hline
\textbf{Request Type} & \textbf{\# Samples} & \textbf{RPS} & \textbf{Avg.} & \textbf{90\%} & \textbf{STD} & \textbf{Error \%} \\ \hline
Insert Timeseries & 71775 & 40.5 & 30 & 41 & 51.71 & 0.00\% \\ \hline
Retrieve Timeseries & 71736 & 40.6 & 23 & 32 & 50.18 & 0.00\% \\ \hline
Insert Grid & 7975 & 4.5 & 91 & 112 & 19.58 & 1.42\% \\ \hline
Retrieve Grid & 7972 & 4.5 & 118 & 165 & 56.15 & 0.00\% \\ \hline
Query Location & 71749 & 40.5 & 3 & 4 & 2.32 & 0.00\% \\ \hline
\textbf{TOTAL} & 310934 & 175.4 & 134 & 503 & 206.40 & 0.04\% \\ \hline
%\multicolumn{4}{l}{$^{\mathrm{a}}$S.D.: Standard Deviation}{$^{\mathrm{b}}$90\%: 90\% percentile}
\end{tabular}
\end{adjustbox}
\label{ptab:obs_all_15_min_summary_throughput}
\end{center}
\end{table}

In all test cases, performance in resolving location-based queries remained the same. This was due to the geo-index build using the NoSQL database. Moreover, as the number of queries was relatively low, those requests were processed in less time compared to other timeseries data.


\begin{table}[tb!]
\caption{Performance while processing 15-min data requests when auto-scaling is enabled.}
\begin{center}
\begin{adjustbox}{width=0.45\textwidth}
\footnotesize
\begin{tabular}{|l|r|r|r|r|r|r|}
\hline
\textbf{Request Type} & \textbf{\# Samples} & \textbf{RPS} & \textbf{Avg.} & \textbf{90\%} & \textbf{STD} & \textbf{Error \%}\\ \hline
Insert Timeseries & 71727 & 40.5 & 34 & 27 & 118.78 & 0.00\% \\ \hline
Retrieve Timeseries & 71693 & 40.5 & 7 & 9 & 18.72 & 0.00\% \\ \hline
Insert Grid & 7968 & 4.5 & 87 & 98 & 14.07 & 0.18\% \\ \hline
Retrieve Grid & 7965 & 4.5 & 89 & 110 & 37.79 & 0.00\% \\ \hline
Query Location & 71704 & 40.5 & 1 & 2 & 2.05 & 0.00\% \\ \hline
\textbf{TOTAL} & 310734 & 175.3 & 130 & 501 & 212.35 & 0.00\% \\ \hline
%\multicolumn{4}{l}{$^{\mathrm{a}}$S.D.: Standard Deviation}{$^{\mathrm{b}}$90\%: 90\% percentile}
\end{tabular}
\end{adjustbox}
\label{ptab:obs_all_auto_15_min_summary_throughput}
\end{center}
\end{table}

\cref{ptab:obs_all_auto_15_min_summary_throughput} shows the performance once the container auto-scaling is enabled. When we were inserting and retrieving data at 15-min resolution, microservices handling grid data had high resource utilization, and 1.4\% of the grid insert requests failed due to timeouts. Once the auto-scaling was enabled, resource utilization of microservices was more uniformly distributed. This reduced both the errors (0.18\%) and the overall latency for inserting and retrieving grid data. From \cref{ptab:obs_all_15_min_summary_throughput} and \cref{ptab:obs_all_auto_15_min_summary_throughput} it can also be seen that 90\% percentile of latency has also reduced due to better workload distribution.

\begin{figure}[!tb]
\centerline{\includegraphics[width=0.5\textwidth]{results/obs/all_auto/obs_all_auto_15m_res_latencies_against_hits.png}}
\caption{Variation of request rate and latency with elapsed time with auto-scaling and 15-min data.}
\label{pfi:test_obs_auto_all_15_min_latency_vs_hits}
\end{figure}

\cref{pfi:test_obs_auto_all_15_min_latency_vs_hits} shows the latency of each request type under 15-min resolution and auto-scaling. Even though the request rate increased linearly (indicated by the red line), no significant increase in latency can be observed for request types other than grid data related requests (indicated by the blue line). Even then, latency was within $2\times$ while the request rate was $5\times$.

%\begin{figure}[!tb]
%\centerline{\includegraphics[width=0.5\textwidth]{results/obs/all_auto/obs_all_auto_15m_transaction_throughtput_vs_threads.png}}
%\caption{Transnational throughput vs number of active threads while performance test with 15min data and auto-scaling enabled.}
%\label{pfi:test_obs_auto_all_15_min_throughput_vs_threads}
%\end{figure}

%\cref{pfi:test_obs_auto_all_15_min_throughput_vs_threads} shows the total server's transaction throughput against the number of active threads.
%The formula for total server transaction throughput is \(<active threads> * 1 second / <1 thread response time>\) \cite{JMeterPluginsTransactionPlugin}. It shows the statistical maximum possible number of transactions based on the number of users accessing the application.
%By combining the \cref{pfi:test_obs_auto_all_15_min_latency_vs_hits}, this graph shows that the throughput of the system gets increased without much change in the latency, thus it proves the scalability of the \acrshort{wdias}.

\begin{figure}[!tb]
    \centering
    \includegraphics[width=0.5\textwidth]{images/obs_all_auto_15m_import_ascii_pods_p1.pdf}
    \caption{Number of running replicas of import ascii grid microservice over time with auto-scaling enabled.}
    \label{pfi:obs_all_auto_15m_import_grid_pod}
\end{figure}

As the performance bottleneck was mostly on microservices handing grid data, in \cref{pfi:obs_all_auto_15m_import_grid_pod}, we look at how the number of \textit{import ascii grid} pods are scheduled with time. In \acrshort{k8s} terminology, a running containerized application is referred to as a \emph{pod}. We configured the \acrshort{k8s} for \textit{import ascii grid} microservice to horizontal scale up to a maximum of ten pods. A new pod was allowed to initiate with a single virtual CPU (vCPU) and was able to scale up to two vCPUs per pod. As shown in the graphs, three pods were scheduled initially as per the configuration of \acrshort{k8s}. As the workload increased, a new pod was spawned when the average CPU utilization reached 80\%. Once the ten pods were reached no new pods were spawned. Pods were also able to scale vertically up to two vCPUs. From the 0.18\% failed insert grid data operations, most of them were caused once the ten pod and 2 vCPUs per pod limit was reached. As seen in  \cref{pfi:obs_all_auto_15m_import_grid_pod}, once the workload is cut off, the system gradually reduces the number of pods and reach the minimum configured number of pods.

%%%%%%%%%%%%%%%%%%%%%%%%%%%%%%%%%%%%%%%%%%%%%%%%
% \subsection{Performance Metrics Analysis}
% \label{psubse:performance_metrics}

%The \emph{Latency} for each operation type kept constant over the whole test plan run time without any significant change. When the request size increased from 24 data points to 96 data points, the latency increase throughout the whole test plan with a smaller number. But for each test run, the latency kept constant over time.
%During the \cref{ptab:obs_all_auto_15_min_summary_throughput} test run, the performance of the Grid data got better when compared to \cref{ptab:obs_all_15_min_summary_throughput}. This means by adding more resources to the \acrshort{wdias}, it can handle more workload on the system.

%While keeping the latency constant without significant change, the \emph{throughput} of the \acrshort{wdias} kept constant while increasing the request size from 60min data (24 data points) to 15min data (96 data points) for all the data types such as Scalar, Vector, and Grid.
%When the number of active threads increased, the \acrshort{wdias} able to provide the same throughput with maintaining the latency stable.

%When looking into the \emph{resource utilization} of \acrshort{wdias}; since it is using the \acrshort{k8s} as the container orchestration system, it allows us to scale up and cool down the system as required based on the workload. This demonstrates the test plan of \cref{ptab:obs_all_15_min_summary_throughput}, and the system gets scale up to the maximum at the peak time. Then cool down to a single pod after finishing the test cases.
%Given above \acrshort{wdias} can run from 1 CPU node to nodes with 100 CPUs. As described in the \cref{pse:wdias_architecture}, it uses many of the concepts of modern microservice architecture to create stateless, failover, redundant microservice to achieve such capabilities.

%The \acrshort{wdias} supports \emph{auto scaling} by out of the box with \acrshort{k8s}. Services can configure with a maximum number of pods to avoid over resource usage. When there is not much workload on the system, the system cools down to fewer pods to save more resources. When there is an issue with a pod, \acrshort{k8s} auto-schedule another pod and remove the unhealthy pod. Also, it allows updating the system without any downtime with rollback updates.

%\emph{Risk of unable to process data} during the test performance, the \acrshort{wdias} processed many requests with higher request size than the normal usage with a lower rate of failures to process the requests, mainly with insert Grid data. If the usage of \acrshort{wdias} wants to reduce the risk of unable to process data, then the system can configure to run with redundant pods to handle spike of workloads. Also while configuring for the auto-scaling, the \acrshort{k8s} can configure to maintain a lower amount of CPU usage such as 50\% to 60\% rather than 80\%. Such configuration with always spawn new pods to handle double of current peak load.

%%%%%%%%%%%%%%%%%%%%%%%%%%%%%%%%%%%%%%%%%%%%%%%%%%%%%%%%%%%%%%%%%%%%%%%%%%%%%%%%
\subsection{Discussion}
\label{psubse:discussion}

As observed from above performance analysis \acrshort{wdias} could handle increasing workloads (in terms of both the request rate and request size) in a salable manner without a significant increase in latency or errors. It was further noted that auto-scaling containers lead to better performance at a reduced cost. \acrshort{wdias} has several other features/properties that is not covered under above analysis. For example, the extension module enables various prepossessing modules to be integrated into the system as plugins.
With large-scale container orchestration, \acrshort{wdias} can be made to scale from a single vCPU node to nodes with 100s of vCPUs. Moreover, due to the microservices architecture, \acrshort{wdias} can create stateless and redundant modules with failover capabilities. For example, when there is an issue with a pod, container orchestration platform could auto-schedule another pod. Furthermore, microservices allow a system to be updated on the fly by deploying update application in parallel to the existing version. Then the new requests are gradually routed to the new application instances without incurring any downtime. If any failures are experienced in the new version, the system can easily rollback to previous version.

For scalar and vector data, \acrshort{wdias} uses a separate InfluxDB database to gain better performance. However, during the experimental setup we did not horizontally scale these databases to avoided potential data consistency issues. Consequently, we observed relatively high variability in insert and retrieval requests due to the large number of read/write operations performed on each database instance. Because of the microservices architecture it is possible to easily extend \acrshort{wdias} to support high availability with timeseries database using sharding and replication features (e.g., InfluxDB Enterprise edition). 
Further the use of \acrshort{netcdf} files affected the performance of grid data related requests. As microservices can be written in any programming language, more computationally heavy modules such as \acrshort{netcdf}-based grid data processing functions could be written using a C or FORTRAN \acrshort{netcdf} wrapper as opposed to our Python \acrshort{netcdf} wrapper.


%%%%%%%%%%%%%%%%%%%%%%%%%%%%%%%%%%%%%%%%%%%%%%%%%%%%%%%%%%%%%%%%%%%%%%%%%%%%%%%%
\section{Summary}
\label{pse:summary}

We presented a weather data integration and assimilation system based on the microservices architecture and container orchestration. As demonstrated by the performance analysis, \acrshort{wdias} can ingest, transform, and export scalar, vector, and grid timeseries data in multiple formats with good throughput and latency characteristics. Moreover, the combination of microservices architecture and container orchestration results in an extensible, scalable, available, and resource efficient system. In addition to adding more extension modules to manipulate timeseries data, in the future, we plan to extend \acrshort{wdias} by introducing publisher-subscriber capabilities for data dissemination and support irregular grid and polygon data. Further, \acrshort{wdias} could be packaged as infrastructure as code, using a tool like Terraform, to simplify the deployment on any Cloud provider. 

%%%%%%%%%%%%%%%%%%%%%%%%%%%%%%%%%%%%%%%%%%%%%%%%%%%%%%%%%%%%%%%%%%%%%%%%%%%%%%%%
\section*{Acknowledgment}
\label{pse:ack}
This research is supported in part by the grant from the Center for Urban Water, Sri Lanka.

%%%%%%%%%%%%%%%%%%%%%%%%%%%%%%%%%%%%%%%%%%%%%%%%%%%%%%%%%%%%%%%%%%%%%%%%%
\graphicspath{ {./images/} }
\newacronym{wdias}{WDIAS}{Weather Data Integration and Assimilation System}

\newacronym{fews}{Delft-FEWS} {Deltares FEWS}
\newacronym{lead}{LEAD}{Linked Environments for Atmospheric Discovery}
\newacronym{dias}{DIAS}{Data Integration and Assimilation System}
\newacronym{madis}{MADIS}{Meteorological Assimilation Data Ingest System}

\newacronym{nwm}{NWMs}{Numerical Weather Models}
\newacronym{soa}{SOA}{Service Oriented Architecture}
\newacronym{wrf}{WRF}{Weather Research and Forecast}
\newacronym{esb}{ESB}{Enterprise Service Bus}
\newacronym{microservice}{Microservice}{Microservice Architecture}

\newacronym{netcdf}{netCDF}{Network Common Data Form}
\newacronym{GRIB}{GRIB}{General Regularly-distributed Information in Binary form}
\newacronym{csv}{CSV}{Comma-separated Values}

\newacronym{rdbms}{RDBMS}{Relational Database Management System}
\newacronym{k8s}{K8s}{Kubernetes}
\newacronym{eks}{Amazon EKS}{Amazon Elastic Kubernetes Service}

\newacronym{go}{Go Lang}{Go Programming Language}
\newacronym{rps}{RPS}{Requests Per Second}
\newacronym{api}{API}{Application Programming Interface}
\newacronym{rps}{RPS}{Request per Second}

\newacronym{curw}{CUrW-SL}{Center for Urban Water, Sri Lanka}

% \printbibliography[title={References}]
\bibliographystyle{IEEEtran}
\bibliography{mendeley}

\end{document}
